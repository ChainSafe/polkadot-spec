\section{Candidate Selection}
\label{sect-primary-validation}

Collators produce candidates as defined in Definition \ref{defn-candidate} and
send network messages to validators
\todo{@fabio}. Validators receive (Algo. \ref{algo-validate-block}) those
messages and consider those for inclusion (Sect.
\ref{sect-primary-validaty-announcement})

\begin{definition}
  \label{defn-candidate}
  A \textbf{candidate}, $C_{coll}(PoV_B)$, is issues by collators and contains the PoV
  block and enough data in order for any validator to verify its validity. A
  candidate is a tuple of the following format:
  \[
  C_{coll}(PoV_B) := (id_p, h_b({B_{^{relay}_{parent}}}), id_{C}, Sig^{Collator}_{SR25519}, Head_p(B), h_b({PoV_B}))
  \]

  where each value represents:
  \begin{itemize}
    \item $id_p$: the Parachain Id this candidate is for.
    \item $h_b({B_{^{relay}_{parent}}})$: the hash of the relay chain block that this
    candidate should be executed in the context of.
    \item $id_C$: the Collator relay-chain account ID as defined in Definition
    \todo{@fabio}.
    \item $Sig^{Collator}_{SR25519}$: the signature on the 256-bit Blake2 hash
    of the block data by the collator.
    \item $Head_p(B)$: the head data of parachain block $B$ as defined in
    Definition \ref{defn-head-data}.
    \item $h_b({PoV_B})$: the 32-byte Blake2 hash of the PoV block.
  \end{itemize}

\end{definition}

\begin{definition}
  \label{defn-head-data}
  The \textbf{head data}, $Head_p(B)$, of a parachain block is a tuple of the following format:
  \[
    (H_i(B), H_p(B), H_r(B))
  \]

  Where $H_i(B)$ is the block number of parachain block $B$, $H_p(B)$ is the
  32-byte Blake2 hash of the parent block header and $H_r(B)$ represents the
  root of the post-execution state.
  \todo{@fabio: clarify if $H_p$ is the hash of the header or full block}
  \todo{@fabio: maybe define those symbols at the start (already defined in the Host spec)?}
\end{definition}

\begin{algorithm}[H]
  \caption[]{\sc PrimaryValidation}
  \label{algo-primary-validation}
  \begin{algorithmic}[1]
    \Require{$B$, $\pi_B$, relay chain parent block $B^{relay}_{parent}$}
    %%  \Ensure{}

    \State Retrieve $v_B$ from the relay chain state at $B^{relay}_{parent}$
    \State Run Algorithm \ref{algo-validate-block} using $B, \pi_B, v_B$
  \end{algorithmic}
\end{algorithm}

\begin{algorithm}[H]
  \caption[]{\sc ValidateBlock}
  \label{algo-validate-block}
  \begin{algorithmic}[1]
    \Require{$B, \pi_B, v_B$}
    %%  \Ensure{}
    \State retrieve the runtime code $R_\rho$ that is specified by $v_B$ from the relay chain state.
    \State check that the initial state root in $\pi_B$ is the one claimed in $v_B$
    \State Execute $R_\rho$ on $B$ using $\pi_B$ to simulate the state.
    \State If the execution fails, return fail.
    \State Else return success, the new header data $h_B$ and the outgoing messages $M$. \todo{@fabio: same as head data?}
  \end{algorithmic}
\end{algorithm}

\section{Candidate Backing}
\label{sect-primary-validaty-announcement}

Validators back the validity respectively the invalidity of candidates by
extending those into candidate receipts as defined in Definition
\ref{defn-candidate-receipt} and communicate it by issuing statements as defined
in Definition \ref{defn-gossip-statement}. Validator $v$ needs to perform
Algorithm \ref{algo-primary-validation-announcement} to announce the result of
primary validation to the Polkadot network. If the validator receiver a
statement from another validator, the candidate is confirmed based on algorithm
\ref{algo-endorse-candidate-receipt}.
\newline

As algorithm \ref{algo-primary-validation-announcement} and
\ref{algo-endorse-candidate-receipt} clarifies, the validator should blacklist
collators which send invalid candidates. If the invalid candidate originated from
another validator, a misbehavior must be announced as defined in \todo{@fabio}.
\newline

The validator tries to back as many candidates as it can, but does not attempt to prioritize
specific candidates. Each validator decides on its own - on whatever metric - which candidate
will ultimately get included in the block.

\begin{definition}
  \label{defn-candidate-receipt}
  A \textbf{candidate receipt}, $C_{receipt}(PoV_B)$, is an extension of a candidate
  as defined in Definition \ref{defn-candidate} which includes additional
  information about the validator which verified the PoV block. The candidate
  receipt is communicated to other validators by issuing a statement as defined
  in Definition \ref{defn-gossip-statement}.
  \newline

  This type is a tuple of the following format:
  \[
  C_{receipt}(PoV_B) := (id_p, h_b({B_{^{relay}_{parent}}}), Head_p(B), id_{C}, Sig^{Collator}_{SR25519}, h_b({PoV_B}), CC(PoV_B))
  \]

  where each value represents:
  \begin{itemize}
    \item $id_p$: the Parachain Id this candidate is for.
    \item $h_b({B_{^{relay}_{parent}}})$: the hash of the relay chain block that this
    candidate should be executed in the context of.
    \item $Head_p(B)$: the head data of parachain block $B$ as defined in Definition \ref{defn-head-data}.
    \todo{@fabio (collator module relevant?)}.
    \item $id_C$: the Collator relay-chain account ID as defined in Definition
    \todo{@fabio}.
    \item $Sig^{Collator}_{SR25519}$: the signature on the 256-bit Blake2 hash
    of the block data by the collator.
    \item $h_b({PoV_B})$: the hash of the PoV block.
    \item $CC(PoV_B)$: Commitments made as a result of validation, as defined in
    Definition \ref{defn-candidate-commitments}.
  \end{itemize}
\end{definition}

\begin{definition}
  \label{defn-candidate-commitments}
  \textbf{Candidate commitments}, $CC(PoV_B)$, are results of the execution and validation of
  parachain (or parathread) candidates whose produced values must be committed
  to the relay chain. A candidate commitments is represented as a tuple of the following format:
  \begin{alignat*}{2}
    CC(PoV_B) &:= (\mathbb{F}, Enc_{SC}(Msg_0, .., Msg_n), H_r(B), \mathbb{U}(R_\rho)) \\
    Msg &:= (\mathbb{O}, Enc_{SC}(b_0,.. b_n))
  \end{alignat*}

  where each value represents:
  \begin{itemize}
    \item $\mathbb{F}$: fees paid from the chain to the relay chain validators.
    \item $Msg$: parachain messages to the relay chain. $\mathbb{O}$ identifies
    the origin of the messages and is a varying data type as defined in
    Definition \todo{@fabio} and can be one of the following values:
    \begin{equation*}
      \mathbb{O} =
      \begin{cases}
        0, & \text{Signed} \\
        1, & \text{Parachain} \\
        2, & \text{Root}
      \end{cases}
    \end{equation*}
    \todo{@fabio: define the concept of "origin"}

    The following SCALE encoded array, $Enc_{SC}(b_0,.. b_n)$, contains the
    raw bytes of the message which varies in size.
    \item $H_r(B)$: the root of a block's erasure encoding Merkle tree
    \todo{@fabio: use different symbol for this?}.
    \item $\mathbb{U}(R_\rho)$: A varying datatype as defined in Definition
    \todo{@fabio} and can contain one of the following values:
    \begin{equation*}
      \mathbb{P} =
      \begin{cases}
        0, & \text{None} \\
        1, & \text{followed by: } R_\rho\\
      \end{cases}
    \end{equation*}
    where $R_\rho$ is a SCALE encoded array containing the new runtime code for
    the parachain. \todo{@fabio: clarify further}
  \end{itemize}
  
\end{definition}

\begin{definition}
  \label{defn-pov-block}
  A \textbf{Gossip PoV block} is a tuple of the following format:
  \[
  (h_b(B_{^{relay}_{parent}}), h_b(candidate), PoV_B)
  \]
  where $h_b(B_{^{relay}_{parent}})$ is the block hash of the relay chain being
  referred to and $h_b(candidate)$ is the hash of some candidate localized to
  the same Relay chain block, whose PoV block is $_b(candidate)$.
\end{definition}

\begin{definition}
  \label{defn-gossip-statement}
  A \textbf{statement} notifies other validators about the validity of a PoV block.
  This type is a tuple of the following format:
  \[
  (Stmt, id_{\mathbb{V}}, Sig^{Valdator}_{SR25519})
  \]
  where $Sig^{Validator}_{SR25519}$ is the signature of the validator and
  $id_{\mathbb{V}}$ refers to the index of validator according to the authority
  set. \todo{@fabio: define authority set (specified in the Host spec)}. $Stmt$
  refers to a statement the validator wants to make about a certain candidate.
  $Stmt$ is a varying datatype as defined in X \todo{@fabio: define} and can be one
  of the following values:

  \begin{equation*}
    Stmt =
    \begin{cases}
      0, & \text{Seconded, followed by: } C_{receipt}(PoV_B) \\
      1, & \text{Validity, followed by: } Blake2(C_{coll}(PoV_B)) \\
      2, & \text{Invalidity, followed by: } Blake2(C_{coll}(PoV_B))
    \end{cases}
  \end{equation*}

  The main semantic difference between `Seconded` and `Valid` comes from the fact
  that every validator may second only one candidate per relay chain block; this
  places an upper bound on the total number of candidates whose validity needs to
  be checked. A validator who seconds more than one parachain candidate per relay
  chain block is subject to slashing.
  \newline

  Validation does not directly create a seconded statement, but is rather upgraded
  by the validator when it choses to back a valid candidate as described in
  Algorithm \ref{algo-primary-validation-announcement}.
\end{definition}

\begin{algorithm}[H]
  \caption[PrimaryValidationAnnouncement]{\sc PrimaryValidationAnnouncement}
  \label{algo-primary-validation-announcement}
  \begin{algorithmic}[1]
    \Require{$PoV_B$}
      \State \textbf{Init} $Stmt$;
      \If {\textsc{ValidateBlock($PoV_B$)} is \textbf{valid}}
        \State $Stmt \leftarrow$ \textsc{SetValid($PoV_B$)}
      \Else
        \State $Stmt \leftarrow$ \textsc{SetInvalid($PoV_B$)}
        \State \textsc{BlacklistCollatorOf}($PoV_B$)
      \EndIf
      \State \textsc{Propagate}($Stmt$)
  \end{algorithmic}
\end{algorithm}

\begin{itemize}
  \item \textsc{ValidateBlock}: Validates $PoV_B$ as defined in Algorithm
  \ref{algo-validate-block}.
  \item \textsc{SetValid}: Creates a valid statement as defined in Definition
  \ref{defn-gossip-statement}.
  \item \textsc{SetInvalid}: Creates an invalid statement as defined in
  Definition \ref{defn-gossip-statement}.
  \item \textsc{BlacklistCollatorOf}: blacklists the collator which sent the
  invalid PoV block, preventing any new PoV blocks from being received. The
  amount of time for blacklisting is unspecified.
  \item \textsc{Propagate}: sends the statement to the connected peers.
\end{itemize}

\begin{algorithm}[H]
  \caption[]{\sc ConfirmCandidateReceipt}
  \label{algo-endorse-candidate-receipt}
  \begin{algorithmic}[1]
    \Require{$Stmt_{peer}$}
      \State \textbf{Init} $Stmt$;
      \State $PoV_B \leftarrow$ \textsc{Retrieve}($Stmt_{peer}$)
      \If {\textsc{ValidateBlock($PoV_B$)} is \textbf{valid}}
        \If {\textsc{AlreadySeconded($B^{relay}_{chain}$)}}
          \State $Stmt \leftarrow$ \textsc{SetValid($PoV_B$)}
        \Else
          \State $Stmt \leftarrow$ \textsc{SetSeconded($PoV_B$)}
        \EndIf
      \Else
        \State $Stmt \leftarrow$ \textsc{SetInvalid($PoV_B$)}
        \State \textsc{AnnounceMisbehaviorOf}($PoV_B$)
      \EndIf
      \State \textsc{Propagate}($Stmt$)
  \end{algorithmic}
\end{algorithm}

\begin{itemize}
  \item $Stmt_{peer}$: a statement received from another validator.
  \item \textsc{Retrieve}: Retrieves the PoV block from the statement
  (\ref{defn-gossip-statement}).
  \item \textsc{ValidateBlock}: Validates $PoV_B$ as defined in Algorithm
  \ref{algo-validate-block}.
  \item \textsc{AlreadySeconded}: Verifies if a parachain block has already been
  seconded for the given Relay Chain block. Validators that second more than one
  (1) block per Relay chain block are subject to slashing. More information is
  available in Definition \ref{defn-gossip-statement}.
  \item \textsc{SetValid}: Creates a valid statement as defined in Definition
  \ref{defn-gossip-statement}.
  \item \textsc{SetSeconded}: Creates a seconded statement as defined in
  Definition \ref{defn-gossip-statement}. Seconding a block should ensure that
  the next call to \textsc{AlreadySeconded} reliably affirms this action.
  \item \textsc{SetInvalid}: Creates an invalid statement as defined in
  Definition \ref{defn-gossip-statement}.
  \item \textsc{BlacklistCollatorOf}: blacklists the collator which sent the
  invalid PoV block, preventing any new PoV blocks from being received. The
  amount of time for blacklisting is unspecified.
  \item \textsc{AnnounceMisbehaviorOf}: announces the misbehavior of the
  validator who claimed a valid statement of invalid PoV block as described in
  algorithm \todo{@fabio}.
  \item \textsc{Propagate}: sends the statement to the connected peers.
\end{itemize}

\subsection{Inclusion of candidate receipt on the relay chain}\label{sect-inclusion-of-candidate-receipt}

\todo{@fabio: should this be a subsection?}

\begin{definition}
  \label{defn-para-proposal}
        {\b Parachain Block Proposal}, noted by $P^B_{\rho}$is a candidate receipt for a parachain block $B$ for a parachain $\rho$ along with signatures for at least 2/3 of $\mathcal{V}_\rho$.  %\syed{and no invalidity for it}{TBS how to deal with message of invalidity announcement}. We still include the proposal if a parachain validator attests to its invalidity, Any primary invalidity attestation or late primary validity attestation are handled just like secondary attestations.

\end{definition}

A block producer which observe a Parachain Block Proposal as defined in definition \ref{defn-para-proposal} \syed{may/should}{?} include the proposal in the block they are producing according to Algorithm \ref{algo-include-parachain-proposal} during block production procedure.

\begin{algorithm}[H]
  \caption[]{\sc IncludeParachainProposal($P^B_{\rho}$)}
  \label{algo-include-parachain-proposal}
  \begin{algorithmic}[1]
    \Require{}
    %%  \Ensure{}

    \State TBS
  \end{algorithmic}
\end{algorithm}

\section{PoV distribution}


\documentclass{book}
\usepackage{geometry,amsmath,amssymb,hyperref,calc,ifthen,alltt,theorem,capt-of,makeidx}
\usepackage{algorithm}
\usepackage{algpseudocode}
%%\usepackage{algorithmic}
\usepackage{xargs}
\usepackage{tikz}

\makeindex
\geometry{letterpaper}

\begin{document}
\title{
    {\Huge Polkadot Networking}\\
    {\Large Protocol Specification}
}

\date{}
\maketitle
\tableofcontents
\newpage

\section{Introduction}

The Polkadot network is decentralized and does not rely on any central authority
or entity in order to achieve a its fullest potential of provided functionality.
Each node with the network can authenticate itself and its peers by using
cryptographic keys, including establishing fully encrypted connections. The
networking protocol is based on the open and standardized \verb|libp2p|
protocol, including the usage of the distributed Kademlia hash table for peer
discovery.

\subsection{Discovery mechanism}

The Polkadot Host uses varies mechanism to find peers within the network, to
establish and maintain a list of peers and to share that list with other peers
from the network.
\newline

The Polkadot Host uses various mechanism for peer dicovery.

\begin{itemize}
    \item Bootstrap nodes - hard-coded node identities and addresses provided by
    network configuration itself. Those addresses are selected an updated by the
    developers of the Polkadot Host. Node addresses should be selected based on
    a reputation metric, such as reliability and uptime.
    \item mDNS - performs a broadcast to the local network. Nodes that might be
    listing can respond the the broadcast.
    \item Kademlia requests - Kademlia supports \verb|FIND_NODE| requests, where
    nodes respond with their list of available peers.
\end{itemize}

\end{document}
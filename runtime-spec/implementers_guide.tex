\newcommand{\todo}[1]{\textcolor{red}{TODO: #1}}

\chapter{Implementer's Guide}

\section{Ramble / Preamble}

This document aims to describe the purpose, functionality, and implementation of
a host for Polkadot's parachains. It is not for the implementor of a specific
parachain but rather for the implementor of the Parachain Host, which provides
security and advancement for constituent parachains. In practice, this is for
the implementors of Polkadot.
\newline

There are a number of other documents describing the research in more detail.
All referenced documents will be linked here and should be read alongside this
document for the best understanding of the full picture. However, this is the
only document which aims to describe key aspects of Polkadot's particular
instantiation of much of that research down to low-level technical details and
software architecture.

\section{Origins}

Parachains are the solution to a problem. As with any solution, it cannot be
understood without first understanding the problem. So let's start by going over
the issues faced by blockchain technology that led to us beginning to explore
the design space for something like parachains.

\subsection{Issue 1: Scalability}

It became clear a few years ago that the transaction throughput of simple
Proof-of-Work (PoW) blockchains such as Bitcoin, Ethereum, and myriad others was
simply too low. \todo{PoS, sharding, what if there were more blockchains, etc.
etc.}
\newline

Proof-of-Stake (PoS) systems can accomplish higher throughput than PoW
blockchains. PoS systems are secured by bonded capital as opposed to spent
effort - liquidity opportunity cost vs. burning electricity. The way they work
is by selecting a set of validators with known economic identity who lock up
tokens in exchange for earning the right to "validate" or participate in the
consensus process. If they are found to carry out that process wrongly, they
will be slashed, meaning some or all of the locked tokens will be burned. This
provides a strong disincentive in the direction of misbehavior.
\newline

Since the consensus protocol doesn't revolve around wasting effort, block times
and agreement can occur much faster. Solutions to PoW challenges don't have to
be found before a block can be authored, so the overhead of authoring a block is
reduced to only the costs of creating and distributing the block.
\newline

However, consensus on a PoS chain requires full agreement of 2/3+ of the
validator set for everything that occurs at Layer 1: all logic which is carried
out as part of the blockchain's state machine. This means that everybody still
needs to check everything. Furthermore, validators may have different views of
the system based on the information that they receive over an asynchronous
network, making agreement on the latest state more difficult.
\newline

Parachains are an example of a \textbf{sharded} protocol. Sharding is a concept
borrowed from traditional database architecture. Rather than requiring every
participant to check every transaction, we require each participant to check
some subset of transactions, with enough redundancy baked in that byzantine
(arbitrarily malicious) participants can't sneak in invalid transactions - at
least not without being detected and getting slashed, with those transactions
reverted.
\newline

Sharding and Proof-of-Stake in coordination with each other allow a parachain
host to provide full security on many parachains, even without all participants
checking all state transitions.

\todo{note about network effects & bridging}

\subsection{Issue 2: Flexibility / Specialization}

"dumb" VMs don't give you the flexibility. Any engineer knows that being able to
specialize on a problem gives them and their users more leverage. \todo{...}
\newline

Having recognized these issues, we set out to find a solution to these problems,
which could allow developers to create and deploy purpose-built blockchains
unified under a common source of security, with the capability of
message-passing between them; a heterogeneous sharding solution, which we have
come to know as \textbf{Parachains}.

\section{Parachains: Basic Functionality}

This section aims to describe, at a high level, the architecture, actors, and Subsystems involved in the implementation of parachains. It also illuminates certain subtleties and challenges faced in the design and implementation of those Subsystems. Our goal is to carry a parachain block from authoring to secure inclusion, and define a process which can be carried out repeatedly and in parallel for many different parachains to extend them over time. Understanding of the high-level approach taken here is important to provide context for the proposed architecture further on.

The Parachain Host is a blockchain, known as the relay-chain, and the actors which provide security and inputs to the blockchain.

First, it's important to go over the main actors we have involved in the parachain host.

\begin{enumerate}
    \item Validators. These nodes are responsible for validating proposed parachain blocks. They do so by checking a Proof-of-Validity (PoV) of the block and ensuring that the PoV remains available. They put financial capital down as "skin in the game" which can be slashed (destroyed) if they are proven to have misvalidated.
    \item Collators. These nodes are responsible for creating the Proofs-of-Validity that validators know how to check. Creating a PoV typically requires familiarity with the transaction format and block authoring rules of the parachain, as well as having access to the full state of the parachain.
    \item Fishermen. These are user-operated, permissionless nodes whose goal is to catch misbehaving validators in exchange for a bounty. Collators and validators can behave as Fishermen too. Fishermen aren't necessary for security, and aren't covered in-depth by this document.
\end{enumerate}

This alludes to a simple pipeline where collators send validators parachain blocks and their requisite PoV to check. Then, validators validate the block using the PoV, signing statements which describe either the positive or negative outcome, and with enough positive statements, the block can be noted on the relay-chain. Negative statements are not a veto but will lead to a dispute, with those on the wrong side being slashed. If another validator later detects that a validator or group of validators incorrectly signed a statement claiming a block was valid, then those validators will be slashed, with the checker receiving a bounty.

However, there is a problem with this formulation. In order for another validator to check the previous group of validators' work after the fact, the PoV must remain available so the other validator can fetch it in order to check the work. The PoVs are expected to be too large to include in the blockchain directly, so we require an alternate data availability scheme which requires validators to prove that the inputs to their work will remain available, and so their work can be checked. Empirical tests tell us that many PoVs may be between 1 and 10MB during periods of heavy load.

Here is a description of the Inclusion Pipeline: the path a parachain block (or parablock, for short) takes from creation to inclusion:

\begin{enumerate}
    \item Validators are selected and assigned to parachains by the Validator Assignment routine.
    \item A collator produces the parachain block, which is known as a parachain candidate or candidate, along with a PoV for the candidate.
    \item The collator forwards the candidate and PoV to validators assigned to the same parachain via the Collation Distribution Subsystem.
    \item The validators assigned to a parachain at a given point in time participate in the Candidate Backing Subsystem to validate candidates that were put forward for validation. Candidates which gather enough signed validity statements from validators are considered "backable". Their backing is the set of signed validity statements.
    \item A relay-chain block author, selected by BABE, can note up to one (1) backable candidate for each parachain to include in the relay-chain block alongside its backing. A backable candidate once included in the relay-chain is considered backed in that fork of the relay-chain.
    \item Once backed in the relay-chain, the parachain candidate is considered to be "pending availability". It is not considered to be included as part of the parachain until it is proven available.
    \item In the following relay-chain blocks, validators will participate in the Availability Distribution Subsystem to ensure availability of the candidate. Information regarding the availability of the candidate will be noted in the subsequent relay-chain blocks.
    \item Once the relay-chain state machine has enough information to consider the candidate's PoV as being available, the candidate is considered to be part of the parachain and is graduated to being a full parachain block, or parablock for short.
\end{enumerate}

Note that the candidate can fail to be included in any of the following ways:

\begin{itemize}
    \item The collator is not able to propagate the candidate to any validators assigned to the parachain.
    \item The candidate is not backed by validators participating in the Candidate Backing Subsystem.
    \item The candidate is not selected by a relay-chain block author to be included in the relay chain.
    \item The candidate's PoV is not considered as available within a timeout and is discarded from the relay chain.
\end{itemize}

This process can be divided further down. Steps 2 & 3 relate to the work of the collator in collating and distributing the candidate to validators via the Collation Distribution Subsystem. Steps 3 & 4 relate to the work of the validators in the Candidate Backing Subsystem and the block author (itself a validator) to include the block into the relay chain. Steps 6, 7, and 8 correspond to the logic of the relay-chain state-machine (otherwise known as the Runtime) used to fully incorporate the block into the chain. Step 7 requires further work on the validators' parts to participate in the Availability Distribution Subsystem and include that information into the relay chain for step 8 to be fully realized.

This brings us to the second part of the process. Once a parablock is considered available and part of the parachain, it is still "pending approval". At this stage in the pipeline, the parablock has been backed by a majority of validators in the group assigned to that parachain, and its data has been guaranteed available by the set of validators as a whole. Once it's considered available, the host will even begin to accept children of that block. At this point, we can consider the parablock as having been tentatively included in the parachain, although more confirmations are desired. However, the validators in the parachain-group (known as the "Parachain Validators" for that parachain) are sampled from a validator set which contains some proportion of byzantine, or arbitrarily malicious members. This implies that the Parachain Validators for some parachain may be majority-dishonest, which means that secondary checks must be done on the block before it can be considered approved. This is necessary only because the Parachain Validators for a given parachain are sampled from an overall validator set which is assumed to be up to <1/3 dishonest - meaning that there is a chance to randomly sample Parachain Validators for a parachain that are majority or fully dishonest and can back a candidate wrongly. The Approval Process allows us to detect such misbehavior after-the-fact without allocating more Parachain Validators and reducing the throughput of the system. A parablock's failure to pass the approval process will invalidate the block as well as all of its descendents. However, only the validators who backed the block in question will be slashed, not the validators who backed the descendents.

The Approval Process looks like this:

\begin{enumerate}
    \item Parablocks that have been included by the Inclusion Pipeline are pending approval for a time-window known as the secondary checking window.
    \item During the secondary-checking window, validators randomly self-select to perform secondary checks on the parablock.
    \item These validators, known in this context as secondary checkers, acquire the parablock and its PoV, and re-run the validation function.
    \item The secondary checkers submit the result of their checks to the relay chain. Contradictory results lead to escalation, where even more secondary checkers are selected and the secondary-checking window is extended.
    \item At the end of the Approval Process, the parablock is either Approved or it is rejected. More on the rejection process later.
\end{enumerate}

These two pipelines sum up the sequence of events necessary to extend and acquire full security on a Parablock. Note that the Inclusion Pipeline must conclude for a specific parachain before a new block can be accepted on that parachain. After inclusion, the Approval Process kicks off, and can be running for many parachain blocks at once.

Reiterating the lifecycle of a candidate:

\begin{enumerate}
    \item Candidate: put forward by a collator to a validator.
    \item Seconded: put forward by a validator to other validators.
    \item Backable: validity attested to by a majority of assigned validators.
    \item Backed: Backable & noted in a fork of the relay-chain.
    \item Pending availability: Backed but not yet considered available.
    \item Included: Backed and considered available.
    \item Accepted: Backed, available, and undisputed
\end{enumerate}

\todo{Diagram: Inclusion Pipeline & Approval Subsystems interaction}

It is also important to take note of the fact that the relay-chain is extended by BABE, which is a forkful algorithm. That means that different block authors can be chosen at the same time, and may not be building on the same block parent. Furthermore, the set of validators is not fixed, nor is the set of parachains. And even with the same set of validators and parachains, the validators' assignments to parachains is flexible. This means that the architecture proposed in the next chapters must deal with the variability and multiplicity of the network state.

\begin{verbatim}
....... Validator Group 1 ..........
.                                  .
.         (Validator 4)            .
.  (Validator 1) (Validator 2)     .
.         (Validator 5)            .
.                                  .
..........Building on C  ...........        ........ Validator Group 2 ...........
        +----------------------+           .                                    .
        |    Relay Block C     |           .           (Validator 7)            .
        +----------------------+           .    ( Validator 3) (Validator 6)    .
                        \                  .                                    .
                            \                 ......... Building on B  .............
                            \
                    +----------------------+
                    |  Relay Block B       |
                    +----------------------+
                                |
                    +----------------------+
                    |  Relay Block A       |
                    +----------------------+
\end{verbatim}

In this example, group 1 has received block C while the others have not due to network asynchrony. Now, a validator from group 2 may be able to build another block on top of B, called C'. Assume that afterwards, some validators become aware of both C and C', while others remain only aware of one.

\begin{verbatim}
....... Validator Group 1 ..........      ........ Validator Group 2 ...........
.                                  .      .                                    .
.  (Validator 4) (Validator 1)     .      .    (Validator 7) (Validator 6)     .
.                                  .      .                                    .
.......... Building on C  ..........      ......... Building on C' .............


....... Validator Group 3 ..........
.                                  .
.   (Validator 2) (Validator 3)    .
.        (Validator 5)             .
.                                  .
....... Building on C and C' .......

        +----------------------+         +----------------------+
        |    Relay Block C     |         |    Relay Block C'    |
        +----------------------+         +----------------------+
                        \                 /
                            \               /
                            \             /
                    +----------------------+
                    |  Relay Block B       |
                    +----------------------+
                                |
                    +----------------------+
                    |  Relay Block A       |
                    +----------------------+
\end{verbatim}

\textbf{Those validators that are aware of many competing heads must be aware of the work happening on each one. They may contribute to some or a full extent on both. It is possible that due to network asynchrony two forks may grow in parallel for some time, although in the absence of an adversarial network this is unlikely in the case where there are validators who are aware of both chain heads.}
\newpage
\section{Transaction}

A transaction is a SCALE encoded array containing varying data types indicating
the resulting Runtime function to be called, including the parameters required
for that function to be executed.
\newline

The first item of the array, $M_i$, is an indicator to the Runtime to which
Polkadot \textit{module}, $m$, the transaction should be forwarded to. The second
item, $F(m)_i$, indicates which \textit{function} (the actual transaction) of the
specified module should be executed.
\newline

$M_i$ is a varying data types pointing to every module exposed to the network.

\[
M_i :=
\begin{cases}
0, & \text{System} \\
1, & \text{Utility} \\
... & \\
7, & \text{Balances} \\
... &
\end{cases}
\]

The $F(m)_i$ indicator varies for each module, since every module has different
functions. As an example, the \verb|Balances| module offers the following
functions:

\[
F(Balances)_i :=
\begin{cases}
0, & \text{transfer} \\
1, & \text{set\_balance} \\
2 & \text{force\_transfer} \\
3 & \text{transfer\_keep\_alive} \\
\end{cases}
\]

As an example, the \verb|transfer| function takes two parameters, 

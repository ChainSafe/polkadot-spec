\newpage
\chapter{Extrinsics}

\section{Introduction}

An extrinsic is a SCALE encoded array consisting of a version number,
signature, and varying data types indicating the resulting Runtime function to
be called, including the parameters required for that function to be executed.
\newline

\section{Preliminaries}

\begin{definition}
    An extrinsic , $tx$, is a tuple consisting of the extrinsic version,
    $T_v$ (Def. \ref{defn-extrinsic-version}), and the body of the extrinsic, $T_b$.

    \[
        tx := (T_v, T_b)
    \]

    The value of $T_b$ varies for each version. The current version 4 is
    described in section \ref{sect-version-four}.
\end{definition}

\begin{definition}
    \label{defn-extrinsic-version}
    $T_v$ is a single byte and defines the extrinsic version. The required
    format of an extrinsic is dictated by the Runtime. Older or unsupported
    version are rejected.
    \newline

    $T_v$ does not consist of the version number directly, rather the version
    number is manipulated with bitwise operators in order to indicate whether
    the extrinsic is signed ("transaction") or unsigned ("inherent").
    \newline

    If the extrinsic is \textbf{signed}, and therefore a transaction, $T_v$ is
    defined as:

    \[
        T_v := n \ | \ 1000 \ 0000
    \]

    where $n$ is the version number and $|$ implies a bitwise OR operator.
    \newline
    \newpage

    If the extrinsic is \textbf{unsigned}, and therefore an inherent, $T_v$ is
    defined as:

    \[
        T_v := n \ \& \ 0111 \ 1111
    \]

    where $n$ is the version number and $\&$ implies a bitwise AND operator.
    \newline

    As an example, for extrinsic format version 4, an signed extrinsic
    represents $T_v$ as \verb|132| while a unsigned extrinsic represents it as
    \verb|4|.
\end{definition}

\section{Extrinsics Body}

\subsection{Version 4}
\label{sect-version-four}

Version 4 of the Polkadot extrinsic format is defined as follows:

\[
    T_b := (A_i, Sig, E, M_i, F_i(m))
\]

where each values represents:
\begin{itemize}
    \item $A_i$: the 32-byte address of the sender (Def. \ref{defn-extrinsic-address}).
    \item $Sig$: the signature of the sender (Def. \ref{defn-extrinsic-signature}).
    \item $E$: the extra data for the extrinsic (Def. \ref{defn-extra-data}).
    \item $M_i$: the indicator of the Polkadot module (Def. \ref{defn-module-indicator}).
    \item $F_i(m)$: the indicator of the function of the Polkadot module (Def. \ref{defn-function-indicator}).
\end{itemize}

\begin{definition}
    \label{defn-extrinsic-address}
    Account Id, $A_i$, is the 32-byte address of the sender of the extrinsic as
    described in the
    \href{https://github.com/paritytech/substrate/wiki/External-Address-Format-(SS58)}{external
    SS58 address format}.
\end{definition}

\begin{definition}
    \label{defn-extrinsic-signature}
    The signature, $Sig$, is a varying data type indicating the used signature
    type, followed by the signature created by the extrinsic author. The
    following types are supported:

    \[
        Sig :=
        \begin{cases}
        0, & \text{Ed25519, followed by: } (b_0, ...,b_{63}) \\
        1, & \text{Sr25519, followed by: } (b_0, ...,b_{63}) \\
        2, & \text{Ecdsa, followed by: } (b_0, ...,b_{64})
        \end{cases}
    \]

    Signature types vary in sizes, but each individual type is always fixed-size
    and therefore does not contain a length prefix. \verb|Ed25519| and
    \verb|Sr25519| signatures are 512-bit while \verb|Ecdsa| is 560-bit, where
    the last 8 bits are the recovery ID.
    \newline

    The signature is created by signing payload $P$.

    \begin{equation}
        \begin{aligned}
        P &:= \begin{cases}
            Raw, & \text{if } |Raw| \leq 256\\
            Blake2(Raw), & \text{if } |Raw| > 256\\
        \end{cases}\\
        Raw &:= (M_i, F_i(m), E, R_v, F_v, H_h(G), H_h(B))\\
        \end{aligned}
    \end{equation}

    where each value represents:
    \begin{itemize}
        \item $M_i$: the module indicator (Def. \ref{defn-module-indicator}).
        \item $F_i(m)$: the function indicator of the module
        (Def. \ref{defn-function-indicator}).
        \item $E$: the extra data (Def. \ref{defn-extra-data}).
        \item $R_v$: a UINT32 containing the specification version of \verb|14|.
        \item $F_v$: a UINT32 containing the format version of \verb|2|.
        \item $H_h(G)$: a 32-byte array containing the genesis hash.
        \item $H_h(B)$: a 32-byte array containing the hash of the block which
        starts the mortality period, as described in Definition
        \ref{defn-extrinsic-mortality}.
    \end{itemize}
\end{definition}

\begin{definition}
    \label{defn-extra-data}
    Extra data, $E$, is a tuple containing additional meta data about the
    extrinsic and the system it is meant to be executed in.

    \[
        E := (T_m, N, P_t)
    \]

    where each value represents:
    \begin{itemize}
        \item $T_m$: contains the SCALE encoded mortality of the extrinsic (Def.
        \ref{defn-extrinsic-mortality}).
        \item $N$: a compact integer containing the nonce of the sender. The
        nonce must be incremented by one for each extrinsic created, otherwise
        the Polkadot network will reject the extrinsic.
        \item $P_t$: a compact integer containing the transactor pay including tip.
    \end{itemize}

\end{definition}

\begin{definition}
    \label{defn-extrinsic-mortality}
    Extrinsic \textbf{mortality} is a mechanism which ensures that an extrinsic
    is only valid within a certain period of the ongoing Polkadot lifetime.
    Extrinsics can also be immortal, as clarified in Section
    \ref{sect-mortality-encoding}.
    \newline

    The mortality mechanism works with two related values:

    \begin{itemize}
        \item $M_{per}$: the period of validity in terms of block numbers from
        the block hash specified as $H_h(B)$ in the payload (Def.
        \ref{defn-extrinsic-signature}). The period must be the power of two,
        such as \verb|32|, \verb|64|, \verb|128|, etc.
        \item $M_{pha}$: the phase in the period that this extrinsic's lifetime
        begins. This value is calculated with a formula and validators can use
        this value in order to determine which block hash is included in the
        payload.
    \end{itemize}

    The extrinsic author uses the number of the block which starts the mortality
    period, $H_i(B)$, and specifies the desired period in order the calculate
    the phase:

    \[
        M_{pha} = H_i(B)\ mod\ M_{per}
    \]

    $M_{per}$ and $M_{pha}$ are then included in the extrinsic, as clarified in
    Definition \ref{defn-extra-data}, in the SCALE encoded form of $T_m$ (Sect.
    \ref{sect-mortality-encoding}). Polkadot validators can use $M_{pha}$
    to figure out the block hash included in the payload, which will therefore
    result in a valid signature if the extrinsic is within the specified period 
    or an invalid signature if the extrinsic "died".

    \subsubsection*{Example}

    The extrinsic author choses $M_{per} = 256$ at block \verb|10'000|,
    resulting with $M_{pha} = 16$. The extrinsic is then valid for blocks
    ranging from \verb|10'000| to \verb|10'256|.

    \subsubsection*{Encoding}\label{sect-mortality-encoding}

    $T_m$ refers to the SCALE encoded form of type $M_{per}$ and $M_{pha}$.
    $T_m$ is the size of two bytes if the extrinsic is considered mortal,
    or simply one bytes with the value equal to zero if the extrinsic is
    considered immortal.

    \[
        T_m := Enc_{SC}(M_{per}, M_{pha})
    \]

    The SCALE encoded representation of mortality $T_m$ deviates from most
    other types, as it's specialized to be the smallest possible value, as
    described in Algorithm \ref{algo-encode-mortality} and
    \ref{algo-decode-mortality}.

    \begin{algorithm}[H]
        \caption[]{\sc Encode Mortality}
        \label{algo-encode-mortality}
        \begin{algorithmic}[1]
            \Require{$M_{per}, M_{pha}$}
            \Statex // If the extrinsic is immortal, specify
            \Statex // a single byte with the value equal to zero.
            \State $\vcenter{
                \begin{flalign*}
                    \Return & 
                    \begin{cases}
                    0 & if\ extrinsic\ is\ immortal 
                    \end{cases}&
                \end{flalign*}
            }$
            \State \textbf{Init} $factor = \textsc{Max}(M_{per} >> 12, 1)$
            \State \textbf{Init} $left = \textsc{Min}(\textsc{Max}(\textsc{TZ}(M_{per})-1,\ 1),\ 15)$
            \State \textbf{Init} $right = \frac{M_{pha}}{factor} << 4$
            \Statex
            \Statex // Returns a two byte value
            \State \Return $left|right$
        \end{algorithmic}
    \end{algorithm}

    \begin{algorithm}[H]
        \caption[]{\sc Decode Mortality}
        \label{algo-decode-mortality}
        \begin{algorithmic}[1]
            \Require{$T_m$}
            \State $\vcenter{
                \begin{flalign*}
                    \Return & 
                    \begin{cases}
                    \textit{Immortal} & if\ T^{b0}_{mort} = 0
                    \end{cases}&
                \end{flalign*}
            }$
            \Statex 
            \State \textbf{Init} $enc = T^{b0}_{mort} + (T^{b1}_{mort} << 8)$
            \State \textbf{Init} $M_{per} = 2 << (enc\ mod\ (1 << 4))$
            \State \textbf{Init} $factor = \textsc{Max}(M_{per} >> 12, 1)$
            \State \textbf{Init} $M_{pha} = (enc >> 4) * factor$
            \State $\vcenter{
                \begin{flalign*}
                    \Return & 
                    \begin{cases}
                    (M_{per}, M_{pha}) & if\ M_{per} \geq 4\ \textbf{and}\ M_{pha} < M_{per} \\
                    Error & else
                    \end{cases}&
                \end{flalign*}
            }$
        \end{algorithmic}
    \end{algorithm}

    \begin{itemize}
        \item $T^{b0}_{mort}$: the first byte of $T_m$.
        \item $T^{b1}_{mort}$: the second byte of $T_m$.
        \item {\sc Min(..)}: returns the minimum of two values.
        \item {\sc Max(..)}: returns the maximum of two values.
        \item {\sc TZ($num$)}: returns the number of trailing zeros in the
        binary representation of $num$. For example, the binary
        representation of \verb|40| is \verb|0010 1000|, which has three
        trailing zeros.
        \item $>>$: performs a binary right shift operation.
        \item $<<$: performs a binary left shift operation.
        \item $|$ : performs a bitwise OR operation.
    \end{itemize}
\end{definition}

\begin{definition}
    \label{defn-module-indicator}
    $M_i$ is an indicator for the Runtime to which Polkadot \textit{module},
    $m$, the extrinsic should be forwarded to.
    \newline

    $M_i$ is a varying data type pointing to every module exposed to the
    network.

    \[
    M_i :=
    \begin{cases}
    0, & \text{System} \\
    1, & \text{Utility} \\
    ... & \\
    7, & \text{Balances} \\
    ... &
    \end{cases}
    \]
\end{definition}

\begin{definition}
    \label{defn-function-indicator}
    $F_i(m)$ is a tuple which contains an indicator, $m_i$, for the Runtime to
    which \textit{function} within the Polkadot \textit{module}, $m$, the
    extrinsic should be forwarded to. This indicator is followed by the
    concatenated and SCALE encoded parameters of the corresponding function,
    $params$.

    \[
        F_i(m) := (m_i, params)
    \]

    The value of $m_i$ varies for each Polkadot module, since every module
    offers different functions. As an example, the \verb|Balances| module has
    the following functions:

    \[
        Balances_i :=
        \begin{cases}
        0, & \text{transfer} \\
        1, & \text{set\_balance} \\
        2 & \text{force\_transfer} \\
        3 & \text{transfer\_keep\_alive} \\
        \end{cases}
    \]
\end{definition}

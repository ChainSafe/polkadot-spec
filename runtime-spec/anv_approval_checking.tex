\section{The Approval Check}
Once a validator has a VRF which tells them to check a block, they announce this
VRF and attempt to obtain the block. It is unclear yet whether this is best done
by requesting the PoV block from parachain validators or by announcing that they
want erasure-encoded chunks.

\subsubsection{Retrieval}
\label{sect-retrieval-of-erasure-chunks}
There are two fundamental ways to retrieve a parachain block for checking
validity. One is to request the whole block from any validator who has attested
to its validity or invalidity. Assigned appoval checker $v$ sends
RequestWholeBlock message specified in Definition
\ref{defn-msg-request-whole-block} to \syed{}{any/all} parachain validator in
order to receive the specific parachain block. Any parachain validator receiving
must reply with PoVBlockRespose message defined in Definition
\ref{defn-pov-block-response}

\begin{definition}
  \label{defn-pov-block-response}
        {\b PoV Block Respose} Message TBS
\end{definition}

The second method is to retrieve enough erasure-encoded chunks to reconstruct the
block from them. In the latter cases an announcement of the form specified in
Definition has to be gossiped to all validators indicating that one needs the
erasure-encoded chunks.

\begin{definition}
  \label{defn-erasure-coded-chunks-request}
        {\b Erasuree-coded chunks request message} TBS
\end{definition}

On their part, when a validator  receive a erasuree-coded chunks request message
it response with the message specified in Definition
\ref{defn-erasure-coded-chunks-response}.

\begin{definition}
  \label{defn-erasure-coded-chunks-response}
        {\b Erasuree-coded chunks response message} TBS
\end{definition}

Assigned appoval checker $v$ must retrieve enough erasure-encoded chunks of the block
they are verifying to be able to reconstruct the block and the erasure chunks 
tree.

%%To that aim the %We never consider substructure of $B$ to be meaningful, so
%$V$ must {\em retrieve} the full {\em candidate proof-of-validity blob}
%$\bar{B}$ before checking.  Now $V$ knows which which nodes have their
%individual chunks, thanks to their availability announcements.  It thus follows
%from our 2/3rd honest assumption that $V$ could always reconstruct $\bar{B}$ by
%obtaining enough chunks $\chunks_B$ from nodes known to posses them.

%We note however that $V$ also knows that all chunks are known by the
%preliminary backing validity checkers aka parachain validators who approved
%$\bar{B}$, as well as approval checkers who already approved $\bar{B}$.  So $V$
%could first contact some node that possesses all of $\bar{B}$, and only then
%begin a full reconstruction process.

%In both cases, $V$ must recompute $\chunks_B$ to verify $\receipt_{B,\cdot}$.
%We therefore cannot see much computational difference between $V$
%reconstructing $\chunks_B$ from arbitrary chunks or from $\bar{B}$ itself.  It
%remains plausible $V$ avoids some networking overhead by asking for $\bar{B}$
%though.  We think a first implementation could reasonably target reconstructing
%$\chunks_B$ from arbitrary chunks, while leaving requests for the full $\bar{B}$
%to future optimisations.

%Ideally $V$ might retrieve the chunks in $\chunks_B$ only using its existing
%connections in our topology specified above, except these intentionally do not
%include 1/3rd of validators.  Also, $V$ need not connect to any node with all
%of $\chunks_B$.  Yet, $V$ should connect to at least one prachain validators in
%$\vals_\rho$ who ideally should check $B$ first.

%We strongly caution against abandoning approval checkers over topology concerns
%because then adversarial influence over the topology could wreck our assignment
%criteria below.

%In fact, our retrieval component could be engineered to avoid requests
%entirely:  After obtaining $\pi_{V,\cdot}$, another validator $V'$ could simply
%compute its own priority for sending its chunk from $\chunks_B$ to $V$.  We
%caution that doing do might become inefficient, either because $V$ winds up
%rejecting sends, or when many nodes go offline.

\subsubsection{Reconstruction}
\label{}
After receiving $2f+1$ of erasure chunks every assigned approval checker $v$
needs to recreate the entirety of the erasure code, hence every $v$ will run
Algorithm \ref{algo-reconstruct-pov} to make sure that the code is complete and
the subsequently recover the original $\bar{B}$.

\begin{algorithm}[H]
  \caption[Reconstruct-PoV-Erasure]{\sc Reconstruct-PoV-Erasure($S_{Er_B}$)}
  \label{algo-reconstruct-pov-erasure}
  \begin{algorithmic}[1]
  \Require
    $S_{Er_B} := {(e_{j_1}, m_{j_1}),\cdot,(e_{j_k}, m_{j_k}))}$ such that $k > 2f$

  %%  \Ensure{}
    \State $\bar{B} \rightarrow$ {\sc Erasure-Decoder}(${e_{j_1},\cdots, e_{j_k}}$)
    \If {{\sc Erasure-Decoder} {\bf failed}}
        \State {\sc Announce-Failure}
        \State \Return
    \EndIf
    \State $Er_B \rightarrow$ {\sc Erasure-Encoder}($\bar{B}$)
    \If {{\sc Verify-Merkle-Proof}($S_{Er_B}$, $Er_B$) {\bf failed}}
      \State {\sc Announce-Failure}
      \State \Return
    \EndIf
    \State \Return $\bar{B}$
  \end{algorithmic}
\end{algorithm}

\subsection{Verification}
%%Verify
%%\If {{\sc Execute}($R_{\rho}$, $\bar{B}$) {\bf failed}}
%%      \State {\sc Announce-Failure}
%%      \State \Return
%%    \EndIf

Once the parachain bock has been obtained or reconstructed the secondary checker
needs to execute the PoV block. We declare a the candidate receipt as invalid if
one one the following three conditions hold: 1) While reconstructing if the
erasure code does not have the claimed Merkle root, 2) the validation function
says that the PoV block is invalid, or 3) the result  of executing the block is
inconsistent with the candidate receipt on the relay chain.

The procedure is formalized in Algorithm


\begin{algorithm}[H]
  \caption[]{\sc RevalidatingReconstructedPoV}
  \label{algo-revalidating-reconstructed-pov}
  \begin{algorithmic}[1]
    \Require{}
    %%  \Ensure{}

    \State TBS
  \end{algorithmic}
\end{algorithm}

If everything checks out correctly, we declare the block is valid. This means
gossiping an attestation, including a reference that identifies candidate
receipt and our VRF as specified in Definition
\ref{defn-secondary-appoval-attestation}.

\begin{definition}
  \label{defn-secondary-appoval-attestation}
        {\b Secondary approval attetstion message} TBS
\end{definition}


\subsection{Process validity and invalidity messages}
When a Black produced receive a Secondary approval attetstion message, it
execute Algorithm \ref{algo-verify-approval-attestation} to verify the VRF and
may need to judge when enough time has passed.

\begin{algorithm}[H]
  \caption[]{\sc VerifyApprovalAttestation}
  \label{algo-verify-approval-attestation}
  \begin{algorithmic}[1]
    \Require{}
    %%  \Ensure{}

    \State TBS
  \end{algorithmic}
\end{algorithm}

These attestations are included in the relay chain as a transaction specified in

\begin{definition}
  \label{defn-approval-attestation-transaction}
        {\b Approval Attestation Transaction} TBS
\end{definition}

Collators reports of unavailability and invalidty specified in Definition
\todo{Define these messages} also  go onto the relay chain as well in the format
specified in Definition

\begin{definition}
  \label{defn-collator-invalidity-transaction}
        {\b Collator Invalidity Transaction}
        TBS
\end{definition}

\begin{definition}
  \label{defn-collator-unavailability-transaction}
        {\b Collator unavailability Transaction}
\end{definition}

\subsection{Invalidity Escalation}\label{escalation}

When for any candidate receipt, there are attestations for both its validity and
invalidity, then all validators acquire and validate the blob, irrespective of
the assignments from section by executing Algorithm \ref{algo-reconstruct-pov}
and \ref{algo-revalidating-reconstructed-pov}.
\newline

We do not vote in GRANDPA for a chain were the candidate receipt is executed
until its vote is resolved. If we have $n$ validators, we wait for $>2n/3$ of
them to attest to the blob and then the outcome of this vote is one of the
following:
\newline

If $>n/3$ validators attest to the validity of the blob and $\leq n/3$ attest to
its invalidity, then we can vote on the chain in GRANDPA again and slash
validators who attested to its invalidity.
\newline

If $>n/3$ validators attest to the invalidity of the blob and $\leq n/3$ attest
to its validity, then we consider the blob as invalid. If the rely chain block
where the corresponding candidate receipt was executed was not finalised, then
we never vote on it or build on it. We slash the validators who attested to its
validity.
\newline

If $>n/3$ validators attest to the validity of the blob and $>n/3$ attest to its
invalidity then we consider the blob to be invalid as above but we do not slash
validators who attest either way. We want to leave a reasonable length of time
in the first two cases to slash anyone to see if this happens.

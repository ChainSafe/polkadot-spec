\documentclass{book}
\usepackage[english]{babel}
\usepackage{geometry,amsmath,amssymb,hyperref,makeidx}
\makeindex
\geometry{letterpaper}

%%%%%%%%%% Start TeXmacs macros
\newcommand{\assign}{:=}
\newcommand{\glossaryentry}[3]{\item[{#1}\hfill]#2\dotfill#3}
\newcommand{\tmop}[1]{\ensuremath{\operatorname{#1}}}
\newcommand{\tmrsub}[1]{\ensuremath{_{\textrm{#1}}}}
\newcommand{\tmtextbf}[1]{{\bfseries{#1}}}
\newcommand{\tmtextit}[1]{{\itshape{#1}}}
\newcommand{\tmtextsc}[1]{{\scshape{#1}}}
\newcommand{\tmtextsf}[1]{{\sffamily{#1}}}
\newcommand{\tmtexttt}[1]{{\ttfamily{#1}}}
\newenvironment{theglossary}[1]{\begin{list}{}{\setlength{\labelwidth}{6.5em}\setlength{\leftmargin}{7em}\small} }{\end{list}}
%%%%%%%%%% End TeXmacs macros

\providecommand{\cdummy}{{\cdot}}
\providecommand{\nobracket}{}
\providecommand{\nosymbol}{}
\providecommand{\tmem}[1]{\tmtextit{#1}}
\providecommand{\tmname}[1]{\tmtextsc{#1}}
\providecommand{\tmop}[1]{\ensuremath{\mathrm{#1}}}
\providecommand{\tmrsub}[1]{\tmrsub{\ensuremath{\mathrm{#1}}}}
\providecommand{\tmsamp}[1]{\tmtextsf{#1}}
\providecommand{\tmstrong}[1]{\tmtextbf{#1}}
\providecommand{\tmtextbf}[1]{\tmtextbf{#1}}
\providecommand{\tmtextit}[1]{\tmtextit{#1}}
\providecommand{\tmverbatim}[1]{\tmtexttt{#1}}
\newtheorem{definition}{Definition}
\newtheorem{notation}{Notation}
%


\begin{document}

\title{
  The Polkadot Host\\
  {\Large Protocol Specification}
}

\date{April 2, 2020}

\maketitle

{\tableofcontents}

\

\

\

\

\

\

\

\

\

\

\

\

\

\

\

\

\

\

\

\

\

\

\

\chapter*{Glossary}

\begin{theglossary}{gly}
  \glossaryentry{\tmtextbf{$P_n$}}{a path graph or a path of n nodes, can be
  represented by sequences of $(v_1, \ldots, v_n)$ where $e_i = (v_i, v_{i +
  1})$ for $1 \leqslant i \leqslant n - 1$ is the edge which connect $v_i$ and
  $v_{i + 1}$}{\pageref{autolab1}}
  
  \glossaryentry{$\mathbb{B}_n$}{a set of all byte arrays of length
  n}{\pageref{autolab2}}
  
  \glossaryentry{I}{the little-endian representation of a non-negative
  interger, represented as $I = (B_n \ldots B_0)_{256}$}{\pageref{autolab3}}
  
  \glossaryentry{$B$}{a byte array $B = (b_0, b_1, \ldots, b_n)$ such that
  $b_i \assign B_i$}{\pageref{autolab4}}
  
  \glossaryentry{$\tmop{Enc}_{\tmop{LE}}$}{$\begin{array}{lll}
    \mathbb{Z}^+ & \rightarrow & \mathbb{B}\\
    (B_n \ldots B_0)_{256} & \mapsto & (B_{0,} B_1, \ldots_{}, B_n)
  \end{array}$}{\pageref{autolab5}}
  
  \glossaryentry{C, blockchain}{a blockchain C is a directed path
  graph.}{\pageref{autolab6}}
  
  \glossaryentry{Block}{a node of the graph in blockchain C and indicated by
  $B$}{\pageref{autolab7}}
  
  \glossaryentry{Genesis Block}{the unique sink of blockchain
  C}{\pageref{autolab8}}
  
  \glossaryentry{Head}{the source of blockchain C}{\pageref{autolab9}}
  
  \glossaryentry{P}{for any vertex $(B_1, B_2)$ where $B_1 \rightarrow B_2$ we
  say $B_2$ is the parent of $B_1$ and we indicate it by $B_2 \assign P
  (B_1)$}{\pageref{autolab10}}
  
  \glossaryentry{BT, block tree}{is the union of all different versions of the
  blockchain observed by all the nodes in the system such as every such block
  is a node in the graph and $B_1$ is connected to $B_2$ if $B_1$ is a parent
  of $B_2$}{\pageref{autolab11}}
  
  \glossaryentry{PBT, Pruned BT}{Pruned Block Tree refers to a subtree of the
  block tree obtained by eliminating all branches which do not contain the
  most recent finalized blocks, as defined in Definition
  \ref{defn-finalized-block}.}{\pageref{autolab12}}
  
  \glossaryentry{pruning}{}{\pageref{autolab13}}
  
  \glossaryentry{G}{is the root of the block tree and B is one of its
  nodes.}{\pageref{autolab14}}
  
  \glossaryentry{CHAIN(B)}{refers to the path graph from $G$ to $B$ in
  (P)$\tmop{BT}$.}{\pageref{autolab15}}
  
  \glossaryentry{head of C}{defines the head of C to be $B$, formally noted as
  $B \assign$\tmtextsc{Head($C$)}.}{\pageref{autolab16}}
  
  \glossaryentry{$| C |$}{defines he length of $C$as a path
  graph}{\pageref{autolab17}}
  
  \glossaryentry{SubChain($B', B$)}{If $B'$ is another node on
  \tmtextsc{Chain($B$)}, then by \tmtextsc{SubChain($B', B$)} we refer to the
  subgraph of \tmtextsc{Chain($B$)} path graph which contains both $B$ and
  $B'$.}{\pageref{autolab18}}
  
  \glossaryentry{$\mathbb{C}_{B'} ((P) \tmop{BT})$}{is the set of all
  subchains of $(P) \tmop{BT}$ rooted at $B'$.}{\pageref{autolab19}}
  
  \glossaryentry{$\mathbb{C}$}{the set of all chains of $(P) \tmop{BT}$,
  $\mathbb{C}_G ((P) \tmop{BT})$ is denoted by $\mathbb{C}$((P)BT) or simply
  $\mathbb{C}$}{\pageref{autolab20}}
  
  \glossaryentry{LONGEST-CHAIN(BT)}{the maximum chain given by the complete
  order over $\mathbb{C}$}{\pageref{autolab21}}
  
  \glossaryentry{LONGEST-PATH(BT)}{the path graph of $(P) \tmop{BT}$ which is
  the longest among all paths in $(P) \tmop{BT}$ and has the earliest block
  arrival time as defined in Definition
  \ref{defn-block-time}.}{\pageref{autolab22}}
  
  \glossaryentry{DEEPEST-LEAF(BT)}{the head of
  LONGEST-PATH(BT)}{\pageref{autolab23}}
  
  \glossaryentry{StoredValue}{the function retrieves the value stored under a
  specific key in the state storage and is formally defined as
  $\begin{array}{l}
    \mathcal{K} \rightarrow \mathcal{V}\\
    k \mapsto \left\{ \begin{array}{cc}
      v & \text{if (k,v) exists in state storage}\\
      \phi & \tmop{otherwise}
    \end{array} \right.
  \end{array}$. Here $\mathcal{K} \subset \mathbb{B}$ and $\mathcal{V} \subset
  \mathbb{B}$ are respectively the set of all keys and values stored in the
  state storage.}{\pageref{autolab24}}
\end{theglossary}

\begin{thebibliography}{DGKR18}
  \bibitem[Bur19]{burdges_schnorr_2019}Jeff Burdges. {\newblock}Schnorr VRFs
  and signatures on the Ristretto group. {\newblock}Technical Report,
  2019.{\newblock}
  
  \bibitem[Col19]{collet_extremely_2019}Yann Collet. {\newblock}Extremely fast
  non-cryptographic hash algorithm. {\newblock}Technical Report, -,
  \url{http://cyan4973.github.io/xxHash/}, 2019.{\newblock}
  
  \bibitem[DGKR18]{david_ouroboros_2018}Bernardo David, Peter Ga{\v z}i,
  Aggelos Kiayias, and  Alexander Russell. {\newblock}Ouroboros praos: An
  adaptively-secure, semi-synchronous proof-of-stake blockchain. {\newblock}In
  \tmtextit{Annual International Conference on the Theory and Applications of
  Cryptographic Techniques},  pages  66--98. Springer, 2018.{\newblock}
  
  \bibitem[Fou20]{web3.0_technologies_foundation_polkadot_2020}Web3.0~Technologies
  Foundation. {\newblock}Polkadot Genisis State. {\newblock}Technical Report,
  \url{https://github.com/w3f/polkadot-spec/blob/master/genesis-state/},
  2020.{\newblock}
  
  \bibitem[Gro19]{w3f_research_group_blind_2019}W3F~Research Group.
  {\newblock}Blind Assignment for Blockchain Extension. {\newblock}Technical
  {\keepcase{Specification}}, Web 3.0 Foundation,
  \url{http://research.web3.foundation/en/latest/polkadot/BABE/Babe/},
  2019.{\newblock}
  
  \bibitem[JL17]{josefsson_edwards-curve_2017}Simon Josefsson  and  Ilari
  Liusvaara. {\newblock}Edwards-curve digital signature algorithm (EdDSA).
  {\newblock}In \tmtextit{Internet Research Task Force, Crypto Forum Research
  Group, RFC},  volume  8032. 2017.{\newblock}
  
  \bibitem[lab19]{protocol_labs_libp2p_2019}Protocol labs. {\newblock}Libp2p
  Specification. {\newblock}Technical Report, Protocol labs,
  \url{https://github.com/libp2p/specs}, 2019.{\newblock}
  
  \bibitem[LJ17]{liusvaara_edwards-curve_2017}Ilari Liusvaara  and  Simon
  Josefsson. {\newblock}Edwards-Curve Digital Signature Algorithm (EdDSA).
  {\newblock}2017.{\newblock}
  
  \bibitem[Per18]{perrin_noise_2018}Trevor Perrin. {\newblock}The Noise
  Protocol Framework. {\newblock}Technical Report,
  \url{https://noiseprotocol.org/noise.html}, 2018.{\newblock}
  
  \bibitem[SA15]{saarinen_blake2_2015}Markku~Juhani Saarinen  and 
  Jean-Philippe Aumasson. {\newblock}The BLAKE2 cryptographic hash and message
  authentication code (MAC). {\newblock}{\keepcase{RFC}} 7693, -,
  \url{https://tools.ietf.org/html/rfc7693}, 2015.{\newblock}
  
  \bibitem[Ste19]{stewart_grandpa:_2019}Alistair Stewart. {\newblock}GRANDPA:
  A Byzantine Finality Gadget. {\newblock}2019.{\newblock}
  
  \bibitem[Tec19]{parity_technologies_substrate_2019}Parity Technologies.
  {\newblock}Substrate Reference Documentation. {\newblock}Rust
  {\keepcase{Doc}}, Parity Technologies,
  \url{https://substrate.dev/rustdocs/}, 2019.{\newblock}
\end{thebibliography}

\printindex

\end{document}

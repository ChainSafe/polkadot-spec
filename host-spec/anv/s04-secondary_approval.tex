\section{Secondary Approval checking}
\label{sect-approval-checking}
Once a parachain block is acted on we carry the secondary validity/availability
checks as follows. A scheme assigns every validator to one or more PoV blocks to
check its validity, see Section \ref{sect-shot-assignment} for details. An
assigned validator acquires the PoV block (see Section
\ref{sect-obtaining-block}) and checks its validity by comparing it to the
candidate receipt. If validators notices that an equivocation has happened an
additional validity/availability assignments will be made that is described in
Section\ref{sect-equivocation-case}.

\subsection{Approval Checker Assignment}

Validators assign themselves to parachain block proposals as defined in
Definition \ref{defn-para-proposal}. The assignment needs to be random.
Validators use their own VRF to sign the VRF output from the current relay chain
block as described in Section \ref{sect-vrf-comp}. Each validator uses the
output of the VRF to decide the block(s) they are revalidating as a secondary
checker. See Section \ref{one-shot-assigment} for the detail.
\newline

In addition to this assignment some extra validators are assigned to every PoV
block which is descirbed in Section \ref{sect-extra-validation assignment}.

\subsection{VRF computation}
\label{sect-vrf-comp}

Every validator needs to run Algorithm \ref{algo-checker-vrf} for every
Parachain $\rho$ to determines assignments. \todo{Fix this. It is incorrect so
far.}

\begin{algorithm}[H]
  \caption[VRF-for-Approval]{\sc VRF-for-Approval($B$, $z$, $s_k$)}
  \label{algo-checker-vrf}
  \begin{algorithmic}[1]
  \Require

    $B$: the block to be approved

    $z$: randomness for approval assignment

    $s_k$: session secret key of validator planning to participate in approval

    \State $(\pi, d) \leftarrow {VRF}(H_h(B),sk(z))$
    \State \Return $(\pi,d)$
  \end{algorithmic}
\end{algorithm}

Where {\sc VRF} function is defined in \cite{polkadot-crypto-spec}.
\newline

\subsection{One-Shot Approval Checker Assignemnt}
\label{sect-shot-assignment}
Every validator $v$ takes the output of this VRF computed by
\ref{algo-checker-vrf} mod the number of parachain blocks that we were decided
to be available in this relay chain block according to Definition
\ref{defn-available-parablock-proposal} and executed. This will give them the
index of the PoV block they are assigned to and need to check. The procedure is
formalised in \ref{algo-one-shot-assignment}.

\begin{algorithm}[H]
  \caption[]{\sc OneShotAssignment}
  \label{algo-one-shot-assignment}
  \begin{algorithmic}[1]
    \Require{}
    %\Ensure{}

    \State TBS
  \end{algorithmic}
\end{algorithm}

\subsection{Extra Approval Checker Assigment}
\label{sect-extra-validation}
Now for each parachain block, let us assume we want $\#VCheck$ validators to
check every PoV block during the secondary checking. Note that $\#VCheck$ is not
a fixed number but depends on reports from collators or fishermen. Lets us
$\#VDefault$ be the minimum number of validator we want to check the block,
which should be the number of parachain validators plus some constant like $2$.
We set

$$\#VCheck = \#VDefault + c_f * \textrm{total fishermen stake}$$ where $c_f$ is
some factor we use to weight fishermen reports. Reports from fishermen about
this
\newline

Now each validator computes for each PoV block a VRF with the input being the
relay chain block VRF concatenated with the parachain index.
\newline

For every PoV bock, every validator compares $\#VCheck - \#VDefault$ to the
output of this VRF and if the VRF output is small enough than the validator
checks this PoV blocks immediately otherwise depending on their difference waits
for some time and only perform a check if it has not seen $\#VCheck$ checks from
validators who either 1) parachain validators of this PoV block 2) or assigned
during the assignment procedure or 3) had a smaller VRF output than us during
this time.
\newline

More fisherman reports can increase $\#VCheck$ and require new checks. We should
carry on doing secondary checks for the entire fishing period if more are
required. A validator need to keep track of which blocks have $\#VCheck$ smaller
than the number of higher priority checks performed. A new report can make us
check straight away, no matter the number of current checks, or mean that we
need to put this block back into this set. If we later decide to prune some of
this data, such as who has checked the block, then we'll need a new approach
here.

\begin{algorithm}[H]
  \caption[]{\sc OneShotAssignment}
  \label{algo-extra-assignment}
  \begin{algorithmic}[1]
    \Require{}
    %\Ensure{}

    \State TBS
  \end{algorithmic}
\end{algorithm}

\syed{}{\todo{so assignees are not announcing their assignment just the result
of the approval check I assume}}

\subsection{Additional Checking in Case of
Equivocation}\label{sect-equivocation-case} In the case of a relay chain
equivocation, i.e. a validator produces two blocks with the same VRF, we do not
want the secondary checkers for the second block to be predictable. To this end
we use the block hash as well as the VRF as input for secondary checkers VRF. So
each secondary checker is going to produce twice as many VRFs for each relay
chain block that was equivocated. If either of these VRFs is small enough then
the validator is assigned to perform a secondary check on the PoV block. The
process is formalized in Algorithm \ref{algo-equivocation-assigment}

\begin{algorithm}[H]
  \caption[]{\sc EquivocatedAssignment}
  \label{algo-equivocation-assigment}
  \begin{algorithmic}[1]
    \Require{}
    %%  \Ensure{}

    \State TBS
  \end{algorithmic}
\end{algorithm}
